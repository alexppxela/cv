%% rubric.tex --- Example of using CurVe.

%% Copyright (C) 2000-2003 Didier Verna.

%% PRCS: $Id: connaissances.tex,v 1.7 2005-09-21 22:31:35 pepe Exp $

%% Author:        Didier Verna <didier@lrde.epita.fr>
%% Maintainer:    Didier Verna <didier@lrde.epita.fr>
%% Created:       Thu Dec 10 16:04:01 2000
%% Last Revision: Mon Mar 25 19:19:50 2002

%% This file is part of CurVe.

%% CurVe may be distributed and/or modified under the
%% conditions of the LaTeX Project Public License, either version 1.1
%% of this license or (at your option) any later version.
%% The latest version of this license is in
%% http://www.latex-project.org/lppl.txt
%% and version 1.1 or later is part of all distributions of LaTeX
%% version 1999/06/01 or later.

%% CurVe consists of the files listed in the file `README'.

\begin{rubric}{Connaissances}
  %% \entry*[Informatique avanc�e]
  %% Syst�mes experts, Datamining, R�seaux de neurones, Algorithmes
  %% g�n�tiques, Grid computing, Programmation par contraintes
  \entry*[Langages]
  C++ (boost), Java, Shell, UML, Python, C\#, \LaTeX
%%   C, Perl, Python, Pascal, OCaml, Bison/Flex
  \entry*[Outils]
  Boost, Emacs, Visual Studio, Eclipse, Git, SVN, CVS, Rational
  (Purify, Quantify), Bullseye, Google perftools, Office
  \entry*[Syst�mes d'exploitation]
  Linux, Solaris, Windows 2000/XP/7
  \entry*[Langues]
  Anglais: lu, �crit et parl�
  \entry*
  Espagnol scolaire
%%   \entry*[Notions]
%%   ADA 95, Ocaml
\end{rubric}
